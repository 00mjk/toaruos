\documentclass{article}

\usepackage[margin=1in]{geometry}
\usepackage{fancyhdr}
\usepackage{amsmath}
\usepackage{listings}
\usepackage{graphicx}
\usepackage[overlap, CJK]{ruby}

\pagestyle{fancy}

\lhead{CS 242}
\chead{}
\rhead{Kevin Lange}
\lfoot{}
\cfoot{}
\rfoot{}

\begin{document}
\begin{CJK}{UTF8}{goth}

\section*{Final Project Proposal}
\subsection*{\emph{Dynamic Linker/Loader for ELF Binaries}}

\subsection*{Overview}

\subsubsection*{Goal}

To develop a stable, simplistic dynamic linker for ELF binaries and
shared libraries.

\subsubsection*{Purpose}

The purpose of this project is to facilitate understanding of the ELF
binary format and its use in dynamic libraries and executables. Work
from this project will be used to give the とあるOS kernel a working
implementation of dynamic libraries.

\subsubsection*{Scope}

This project will be developed for the とあるOS kernel and/or in
user-space under Linux.

\subsubsection*{Description}

The linker/loader will be built to handle 32-bit ELF binaries
compiled for use with shared libraries. It will load the dynamic
binaries and it associated chain of library dependencies and link
the symbol tables of the binary and libraries to create an
executable process image.

\subsubsection*{Schedule}

The four week schedule for this project will be broken up as follows:

\begin{enumerate}
    \item \texttt{readelf} implementation to facilitate learning the ELF binary format.
    \item Kernel-mode static ELF binary loading and execution.
    \item Loading and execuation of binaries through the linker.
    \item Dynamic linking and execution of binaries and libraries.
\end{enumerate}



\end{CJK}
\end{document}
